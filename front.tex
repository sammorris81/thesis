%% ------------------------------ Abstract ---------------------------------- %%
\begin{abstract}
% Spatial methods are important for extremes
% Key points to our methods: Bayesian inference, low-rank approximations

In recent years, extreme values analysis (EVA) have started to benefit from methods that allow for dependence between observations.
In a spatial setting, this is important for many phenomenon (e.g. ozone, rainfall, wildfires) because spatial methods borrow information across space for informing inference and making predictions at unobserved locations.
When working with dependent extremes, max-stable processes are the analog to Gaussian processes.
These processes are exceptionally flexible, but they can be computationally challenging to work with for high dimensional data.

Some current methods address the dimensionality issues for max-stable processes using pairwise likelihoods, and others circumvent the computational challenges with low-rank approximations to max-stable processes.
While these provide feasible solutions, they can still be very slow to fit, and can be sensitive to distributional assumptions.
We provide contributions to three areas of the analysis of spatial extremes.
First, we develop a space-time \skewt method for threshold exceedances.
In this method, we use a local \skewt process to model asymptotic dependence.
We demonstrate that in most cases the method performs comparably or better than Gaussian processes and max-stable processes.

We develop a max-stable model for modeling rare binary data, and demonstrate its performance relative to spatial probit and logistic methods through a simulation study and data analysis for \emph{Tamarix ramosissima}.
Lastly, we present empirical basis functions (EBF) as a data-driven way to get a low-rank approximation to max-stable processes.
We can use the EBFs as an exploratory method to examine important spatial trends as well as for Bayesian inference and predictions as unobserved locations.
We then demonstrate our method for a data analysis on wildfire data in the state of Georgia as well as precipitation data in the eastern U.S.

\end{abstract}


%% ---------------------------- Copyright page ------------------------------ %%
%% Comment the next line if you don't want the copyright page included.
\makecopyrightpage

%% -------------------------------- Title page ------------------------------ %%
\maketitlepage

%% -------------------------------- Dedication ------------------------------ %%
\begin{dedication}
 \centering To Adam
\end{dedication}

%% -------------------------------- Biography ------------------------------- %%
\begin{biography}
Sam was born on August 26, 1981 in Jacksonville, North Carolina.
As the son of a career marine, 
\end{biography}

%% ----------------------------- Acknowledgements --------------------------- %%
\begin{acknowledgements}
First I would like to thank my partner, Adam for his unwavering support.
I would also like to thank my advisor Dr. Brian Reich for his guidance on this work.
I would like to thank Emeric Thibaud and Daniel Cooley for their insight and suggestion on Chapters 2 and 4.

\end{acknowledgements}


\thesistableofcontents

\thesislistoftables

\thesislistoffigures
