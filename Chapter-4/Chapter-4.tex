\chapter{Empirical Basis Functions for Max-Stable Spatial Dependence}
\label{chap:four}
\section{Introduction}\label{s:intro}

\section{Model}\label{s:model}

Let $Y_{t}(\bs)$ be the observation at spatial location $\bs$ and time $t$.  We temporarily drop the subscript $t$ and describe the model for the process $Y(\bs)$ for a single time point, but return to the spatiotemporal setting in Section \ref{s:estimation}.
To focus attention on the extreme values, we emphasize the statistical model for exceedances above a location-specific threshold $T(\bs)$.
We begin by specifying a spatial model for the complete data $Y(\bs)$ and then use the censored likelihood defined by $T(\bs)$ for inference as described in Section \ref{s:MCMC}.
Although the model presented implements a censored likelihood, the model also can fit uncensored data (such as block-maxima) by setting $T(\bs) = -\infty$.

Spatial dependence is captured by modeling $Y(\bs)$ as a max-stable process (ref).
Max-stable processes have generalized extremal value (GEV; see Appendix A.1) marginal distribution.
The GEV has three parameters: location $\mu(\bs)$; scale $\sigma(\bs)$; and shape $\xi(\bs)$.
Spatial dependence is present both in the GEV parameters but also the standardized residual process
\beq\label{Y2Z}
 Z(\bs) = \left\{1+\frac{\xi(\bs)}{\sigma(\bs)}\left[Y(\bs) - \mu(\bs)\right]\right\}^{1/\xi(\bs)},
\eeq which has unit $\Frechet$ (i.e., GEV with location, scale, and shape all equal one) marginal distribution for all $\bs$.


Our objective is to identify a low-rank model for the spatial dependence of $Z(\bs)$.
The spectral representation theorem (ref) states that any max-stable process can be written
\beq\label{spectral}
  Z(\bs) = \mbox{sup}_l B(\bs,\bt_l)A_{l}
\eeq
where the function $B$ satisfies $B(\bs,\bt)>0$ for all $(\bs,\bt)$ and $\int B(\bs,\bt)d\bt=1$ for all $\bs$, and $(\bt_l,A_l)$ for $l=1,...,\infty$ are a Poisson process with intensity measure $dA d\bt/A^2$.
This representation provides a means of truncation.
Ref propose the max-linear model
\beq\label{spectral}
Z(\bs) = \bigvee_{l=1}^L B_{l}(\bs)A_l
\eeq
where $B_{l}(\bs)>0$, $\int B_l(\bs)d\bs=1$ for all $L$, and $A_l$ are independent $\Frechet$ random variables.


The assumption that $Z(\bs)$ equals exactly the maximum of a small number of functions is unrealistic, especially for data measured with error.
We therefore follow the Reich and Shaby (ref) and decompose  $Z(\bs)$ as $Z(\bs)=\theta(\bs)\varepsilon(\bs)$ where $\theta(\bs)$ is a spatial process and $\varepsilon(\bs)\iid$ GEV$(1,\alpha,\alpha)$ is independent error.
The spatial component is
\beq \label{theta}
  \theta(\bs) = \left(\sum_{l=1}^LB_{l}(\bs)^{1/\alpha}A_{l}\right)^{\alpha}.
\eeq
If $B_{l}(\bs)>0$, $\sum_{l=1}^LB_{l}(\bs)=1$ for all $\bs$, and the $A_{l}$ have positive stable (PS; Appendix A.1) distribution $A_{l}\iid$ PS$(\alpha)$, then $Z(\bs)$ is max-stable and has unit $\Frechet$ marginal distributions.

Extremal spatial dependence can be summarized by the extremal coefficient (EC; ref) $\vartheta(\bs,\bt)\in[1,2]$, where
\beq\label{ECdev}
  \mbox{Prob}[Z(\bs)<c,Z(\bt)<c] = \mbox{Prob}[Z(\bs)<c]^{\vartheta(\bs,\bt)}.
\eeq
For the PS random effects model the EC has the form
\beq\label{EC}
   \vartheta(\bs,\bt) = \sum_{l=1}^L \left[B_{l}(\bs)^{1/\alpha}+B_{l}(\bs)^{1/\alpha}\right]^\alpha.
\eeq
In particular, $\vartheta(\bs,\bs) = 2^{\alpha}$ for all $\bs$.

\section{Estimating the spatial dependence function}\label{s:estimation}

To estimate the extremal coefficient function, we consider the process at $n_s$ spatial locations $\bs_1,...,\bs_{n_s}$ and $n_t$ times $t=1,...,n_t$.
Denote $Y_t(\bs_i) = Y_{it}$, $B_l(\bs_i) = B_{il}$, $T(\bs_i)=T_i$, and $\vartheta(\bs_i,\bs_j) = \vartheta_{ij}$.
In this section we develop an algorithm to estimate the spatial dependence parameter $\alpha$ and the $n_s\times L$ matrix $\bB = \{B_{il}\}$.
Given these parameters, we insert them into our model and proceed with Bayesian analysis as described in Section \ref{s:MCMC}.
Our algorithm has the following steps:
\begin{itemize}
  \item[] (1) Obtain an initial estimate of the extremal coefficient for each pair of locations, ${\hat \vartheta}_{ij}$.
  \item[] (2) Spatially smooth these initial estimates ${\hat \vartheta}_{ij}$ using kernel smoothing to obtain ${\tilde \vartheta}_{ij}$.
  \item[] (3) Estimate the spatial dependence parameters by minimizing the difference between model-based coefficients, $\vartheta_{ij}$, and smoothed coefficients, ${\tilde \vartheta}_{ij}$.
\end{itemize}

The first-stage estimates are obtained using the approach of \hl{citation for pairwise estimates}.
To estimate the spatial dependence we first remove variation in the marginal distribution.
Let $U_{it} = \sum_{k=1}^{n_t} I[Y_{ik}<Y_{it}]/n_t$, so that the $U_{it}$ are approximately uniform at each location.
Then for some extreme probability $q\in(0,1)$, solving (\ref{ECdev}) suggests the estimate
\beq\label{EChat0}
   {\hat \vartheta}_{ij}(q) = \frac{\log[Q_{ij}(q)]}{\log(q)},
\eeq
where $Q_{ij}(q) = \sum_{t=1}^{n_t}I[U_{it}<q,U_{jt}<q]/n_t$ is the sample proportion of the time points at which both sites are less than $q$.
Since all large $q$ give valid estimates, we average over a grid of $q$ with $q_1<...<q_{n_q}$
\beq\label{EChat1}
{\hat \vartheta}_{ij} = \frac{1}{n_q}\sum_{j=1}^{n_q}{\hat \vartheta}_{ij}(q_j).
\eeq

Assuming the true EC is smooth over space, the initial estimates ${\hat \vartheta}_{ij}$ can be improved by smoothing.
Let
\beq\label{EChat2}
  {\tilde \vartheta}_{ij} = \frac{\sum_{u=1}^{n_s}\sum_{v=1}^{n_s} w_{iu}w_{jv}{\hat \vartheta}_{uv}}
  {\sum_{u=1}^{n_s}\sum_{v=1}^{n_s} w_{iu}w_{jv}},
\eeq
where $w_{iu} = \exp[-(||\bs_i-\bs_u'||/\phi)^2])$ is the Gaussian kernel function with bandwidth $\phi$.
The elements ${\hat \vartheta}_{ii}$ do not contribute any information as ${\hat \vartheta}_{ii}=1$ for all $i$ by construction.
To eliminate the influence of these estimates we set $w_{ii}=0$.
However, this approach does give imputed values ${\tilde \vartheta}_{ii}$, which provide information about small-scale spatial variability.

The dependence parameters are estimated by comparing estimates ${\tilde \vartheta}_{ij}$ with the model-based values $\vartheta_{ij}$.
For all $i$, $\vartheta_{ii} = 2^{\alpha}$, and therefore we set $\alpha$ to $\alphahat = \log_2(\sum_{i=1}^{n_s}{\tilde \vartheta}_{ii}/n_s)$.
Given $\alpha=\alphahat$, it remains to estimate $\bB$.
The estimate ${\hat \bB}$ is the minimizer of
\beq\label{Bhat}
\sum_{i<j} \left({\tilde \vartheta}_{ij} - \vartheta_{ij}\right)^2
  =
  \sum_{i<j} \left({\tilde \vartheta}_{ji} - \sum_{l=1}^L[B_{il}^{1/\alphahat} + B_{jl}^{1/\alphahat}]^{\alphahat}\right)^2
\eeq
under the restrictions that $B_{il}\ge 0$ for all $i$ and $l$ and $\sum_{l=1}^LB_{il}=1$ for all $i$.
Since the minimizer of (\ref{Bhat}) does not have a closed form, we use block coordinate descent to obtain ${\hat \bB}$.
We cycle through spatial locations and update the vectors $(B_{i1},...,B_{iL})$ conditioned on the values for the other location and repeat until convergence.
At each step, we use the restricted optimization routine in the {\tt R} (ref) function {\tt optim}.
This algorithm gives estimates of the $B_{il}$ at the $n_s$ data locations, but is easily extended to all $\bs$ for spatial prediction.
The kernel smoothing step ensures that the estimates for ${\hat B}_{il}$ are spatially smooth, and thus interpolation of the ${\hat B}_{il}$ gives spatial functions ${\hat B}_l(\bs)$.

The relative contribution of each term can be measured by
\beq\label{v}
v_l = \frac{1}{n_s}\sum_{i=1}^{n_s}{\hat B}_{il}.
\eeq
Since $\sum_{l=1}^L{\hat B}_{il}=1$ for all $i$, we have $\sum_{l=1}^Lv_l = 1$.
Therefore, terms with large $v_l$ are the most important.
The order of the terms is arbitrary, and so we reorder the terms so that $v_1\ge...\ge v_L$.

\section{Bayesian implementation details}\label{s:MCMC}
For our data analysis in Section \ref{s:analysis} we allow the GEV location and scale parameters, denoted $\mu_{it}$ and scale $\sigma_{it}$ respectively, to vary with space and time.
The GEV shape parameter $\xi$ is held constant over space and time because this parameter is notoriously difficult to estimate (ref).
Collectively, let the marginal GEV parameters at location $i$ and time $t$ be $\Theta_{it} = \{\mu_{it},\sigma_{it},\xi\}$.
The GEV location and scale vary according to covariates $\bX_{it}$ with $\mu_{it} = \bX_{it}^T\bbeta_1$ and
$\mbox{log}(\sigma_{it}) = \bX_{it}^T\bbeta_2$.

As shown in \hl{Reich and Shaby (ref)}, the uncensored responses $Y_{it}$ are conditionally independent given the spatial random effects, with conditional distribution
\beq\label{Ycond}
   Y_{it}|\theta_{it},\Theta_{it}\indep GEV(\mu^*_{it}, \sigma_{it}^*,\xi^*),
\eeq
where $\mu_{it}^* = \mu_{it} + \frac{\sigma_{it}}{\xi}(\theta_{it}^\xi-1)$,
$\sigma_{it}^* = \alpha\sigma_{it}\theta_{it}^\xi$, and $\xi^* = \alpha\xi$.
Therefore, the conditional likelihood conveniently factors across observations; marginalizing over the random effect $\theta_{it}$ induces extremal spatial dependence.
To focus on the extreme values above the local threshold $T_i$, we use the censored likelihood
\beq\label{g}
d(y;\theta_{it},\Theta_{it}, T_i)  =
\left\{\begin{array}{ll}
    F(y;\mu_{it}^*,\sigma_{it}^*,\xi^*) & y \le T_i \\
  f(y;\mu_{it}^*,\sigma_{it}^*,\xi^*) & y>T_i,
\end{array}\right.
\eeq
where $F$ and $f$ are the GEV distribution and density functions, respectively, defined in Appendix A.1.


In summary, given the estimates of $\alpha$ and $\bB$, the hierarchical model is
\beqn \label{bayesmodel}
  Y_{it} |\theta_{ij} & \indep & d(y;\theta_{it},\Theta_{it}, T_i) \\
  \theta_{it} &=& \left(\sum_{l=1}^L{\hat B}_{il}^{1/\alphahat}A_{lt}\right)^{\alphahat}
  \mbox{\ \ \ where \ \ \ }
  A_{lt} \iid PS(\alphahat)\nonumber\\
  \mu_{it} &=& \bX_{it}^T\bbeta_1
  \mbox{\ \ \ and \ \ \ }
  \mbox{log}(\sigma_{it}) = \bX_{it}^T\bbeta_2. \nonumber
\eeqn
To complete the Bayesian model, we select independent normal priors with mean zero and variance $\sigma^2_1$ and $\sigma^2_2$ for the components of $\bbeta_1$ and $\bbeta_2$ respectively, and $\xi\sim \mbox{Normal}(0,0.5^2)$.
We use InvGamma(1, 1) priors for $\sigma^2_1$ and $\sigma^2_2$.
We estimate parameters $\Theta = \left\{A_{lt}, \bbeta_1, \bbeta_2, \xi, \sigma^2_1, \sigma^2_2 \right\}$ using Markov chain Monte Carlo methods.
We use a Metropolis-Hastings algorithm to update the model parameters with random walk candidate distributions for all parameters except $\sigma^2_1$ and  $\sigma^2_2$ which we update using Gibbs sampling.
The PS density is challenging to evaluate as it does not have a closed form.
One technique to avoid this complication is to incorporate auxiliary random variables (Stephenson et al, \hl{ref}), but we opt for a numerical approximation to the integral as described in Appendix \hl{which one}.

The first-stage estimate of the extremal coefficients has three tuning parameters: the quantile thresholds $q_1,...,q_{n_q}$, the kernel bandwidth $\phi$, and the number of terms $L$.
In Section \ref{s:analysis} we explore a few possibilities for $\phi$ and $L$ and discuss sensitivity to these choices.
The second-stage Bayesian analysis requires selecting thresholds $T_i,...,T_{n_s}$.  For this we use spatially smoothed sample quantiles.
That is, we set $T_i$ to the 0.95 quantile of the $Y_{it}$ and $Y_{jt}$ for sites $j$ with $||\bs_i-\bs_j||<r$, where $r$ is set to XXX?

\section{Data analysis}\label{s:analysis}
In this section, we illustrate our method using both points above a threshold and block maxima.
In Section \ref{s:georgia}, we present an analysis using annual acreage burned due to wildfires in Georgia from 1965 -- 2014.
This is followed in Section \ref{s:precip} by an analysis of precipitation data in the eastern U.S.

\subsection{Analysis of extreme Georgia fires}\label{s:georgia}
The dataset used for our application is composed of yearly acreage burned due to wildfires for each county in Georgia from 1965 -- 2014 (\texttt{http://weather.gfc.state.ga.us/FireData/}).
Figure \ref{fig:firets25} shows the time series of $\log$(acres burned) for 25 randomly selected counties.
Based on this plot and other exploratory analysis, we see no evidence of non-linear trends and proceed with linear time trends for the GEV location and scale parameters.
For covariates we use the standardized linear time trend $t^* = (t-n_t/2)/n_t$, and $L$ bivariate Gaussian kernel functions $\widetilde{B}_{il}$, and their interactions: $\bX_{it} = (1,t^*,\widetilde{B}_{i1},...,\widetilde{B}_{iL},t^*\widetilde{B}_{i1},...,t^*\widetilde{B}_{iL})^T$.
For the bivariate Gaussian kernels, we select $L$ knot locations, $\bv_{1}, \ldots, \bv_{L}$ from the county centroids, using a space-filling design (\hl{ref add to bibtex Johnson, M.E., Moore, L.M., and Ylvisaker, D. ,1990}).
Then
\begin{align}
  \widetilde{B}_{il} = \exp\left\{-\frac{||\bs_i - \bv_l||^2}{\rho^2}\right\}
\end{align}
where $\rho$ is included as an unknown parameter with a U($\rho_l$, $\rho_u$) prior where $\rho_l$ is the 5th quantile of $||\bs_i - \bv_l||^2$ and $\rho_u$ is the 95th quantile of $||\bs_i - \bv_l||^2$ \hl{basically I took the 5th and 95th quantile of the squared distances between the sites and the knots}.


\vspace{2em} % remove this

\begin{figure}[htbp]  % markdown/eda/eda-plotting.R
  \centering
  \includegraphics[width=0.47\linewidth]{plots/fire-spag-rand-25}
  \includegraphics[width=0.47\linewidth, trim = 0 10em 0 10em]{plots/fire-spatial-q95.pdf}
  \caption{Time series of log acres burned for 25 randomly selected counties with colors coding the county's quadrant (left), and spatially smoothed threshold values, $T_i$ for each county (right).}
  \label{fig:firets25}
\end{figure}

We estimate the extremal coefficient function $\hat{\theta}_{ij}$ by setting $q_1 = 0.90$ and using $n_q = 100$.
With more data, it would possible to increase $q_1$, but we set $q_1 = 0.90$ to increase the stability when estimating $\hat{\vartheta}_{ij}$.

Because these data are not block-maxima, we select a site-specific threshold $T_i$ to use in the analysis with the following algorithm.
Without some adjustment to the data, it is challenging to borrow information across sites to inform the threshold selection.
We first standardize the data, separated by county, by subtracting the site's median and dividing by the site's interquartile range.
Denote the standardized data by $\widetilde{\bY}_i$.
% \begin{align}
%   \widetilde{\bY}_i = \frac{\bY_i - \text{med}(\bY_i)}{\text{IQR}(\bY_i)}
% \end{align}
% where med$(\cdot)$ is the median, and IQR$(\cdot)$ is the inter-quartile range.
Then we combine all sites together and plot a mean residual plot for $\widetilde{Y}_{it}, i = 1, \ldots, n_s$ and $t = 1, \ldots, n_t$.
The mean residual plot is given in Figure \ref{fig:mrlthresh}.
Based upon the mean residual plot, we select the 95th percentile for the threshold.
To calculate $T_i$ for each county, we use the 95th percentile for the combined data for county $i$ and its five closest counties.

\begin{figure}[htbp]
  \centering
  \includegraphics[width = \linewidth]{plots/fire-mrl-plots.pdf}  % markdown/eda/eda-plotting.R
  \caption{Mean residual plot for the data pooled across counties after standardizing using the county's median and interquartile range. The two panels show different ranges on the x-axis and include a vertical line at the sample 95th percentile.}
  \label{fig:mrlthresh}
\end{figure}

%\begin{figure}[htbp]
%  \centering
%  \includegraphics[width = 0.47\linewidth, trim = 0 10em 0 10em]{plots/fire-spatial-q95.pdf}
%  \caption{Spatially smoothed threshold values for each county.}
%  \label{fig:mrlthresh}
%\end{figure}

\subsection{Results for fire analysis}\label{s:results-fire}
We use 10-fold cross-validation to assess the predictive performance of a model.
For each method, we randomly select 90\% of the observations across counties and years to be used as a training set to fit the model.
The remaining 10\% of sites and years are withheld for testing model predictions.
To assess the predictions for the test set, we use quantile scores and Brier scores \hl{citation}.
The quantile score is given by \hl{give formula}.
The Brier score is given by \hl{give formula}.
For both of these methods, we use a negative orientation, so a lower score indicates a better fit.
The Brier and quantile scores for the fire analysis are given in Table \ref{tbl:firescores}.

Based on the Brier scores and quantiles scores, we run a full analysis using all of the data with $L = 35$.
Posterior summaries for each county's $\beta_\text{time}$ coefficient are given in Figures \ref{fig:ebfpost} and \ref{fig:gskpost}.
These plots both seem to catch similar features with some differences particularly in the posterior distribution of the county-specific $\beta_{\mu, \text{time}}$.

\begin{table}[htbp]
\caption{Average Brier scores ($\times 100$) for selected thresholds and quantile scores for selected quantiles for fire analysis \hl{Need timings for L = 35, 40 to be added in}}
\label{tbl:firescores}
\centering
\small
  \begin{tabular}{lc|rr|rr|c}
  \multicolumn{2}{c}{  }& \multicolumn{2}{|c|}{Brier Scores ($\times 100$)} & \multicolumn{2}{|c|}{Quantile Scores} & Time (in hours)\\
  \hline
  & Process & $q(0.95)$ & $q(0.99)$ & $q(0.95)$ & $q(0.99)$\\
  \hline
  \multirow{2}{*}{L = 5}  & EBF & 5.640 & 2.265 & 135.685 & 80.471 & 1.1\\
                          & GSK & 5.726 & 2.301 & 134.419 & 78.639 & 1.1\\
  \hline
  \multirow{2}{*}{L = 10} & EBF & 5.329 & 2.130 & 127.313 & 75.974 & 1.8\\
                          & GSK & 5.311 & 2.142 & 127.593 & 75.123 & 1.9\\
  \hline
  \multirow{2}{*}{L = 15} & EBF & 4.997 & 2.043 & 128.277 & 68.946 & 2.7\\
                          & GSK & 4.907 & 2.034 & 124.537 & \textbf{59.266} & 2.7\\
  \hline
  \multirow{2}{*}{L = 20} & EBF & 4.930 & 2.036 & 122.394 & 66.413 & 3.7\\
                          & GSK & 4.864 & 2.043 & 121.145 & 62.172 & 3.7\\
  \hline
  \multirow{2}{*}{L = 25} & EBF & 4.776 & \textbf{1.920} & 116.944 & 61.704 & 4.8\\
                          & GSK & 4.740 & 1.921 & 113.872 & 59.524 & 4.7\\
  \hline
  \multirow{2}{*}{L = 30} & EBF & 4.745 & 1.923 & 114.878 & 62.020 & 5.9\\
                          & GSK & 4.719 & 1.936 & 114.918 & 61.905 & 5.7\\
  \hline
  \multirow{2}{*}{L = 35} & EBF & 4.761 & 1.920 & 115.696 & 62.581 & \\
                          & GSK & 4.767 & 1.933 & 114.026 & 60.805 & \\
  \hline
  \multirow{2}{*}{L = 40} & EBF & 4.722 & 1.935 & 115.213 & 62.039 & \\
                          & GSK & \textbf{4.716} & 1.921 & \textbf{113.362} & 60.300 & \\
  \hline
	\end{tabular}
\end{table}

\hl{We need a plot of the first 5 basis functions, and we also talked about the \% variability. I wasn't sure if this should be for $L = 35$ or a different number of knots.}

\begin{figure}  % markdown/fire-analysis/combine-tables.R
  \centering
  \includegraphics[width=\linewidth]{plots/ebf-post-betatime.pdf}
  \caption{Posterior summaries of $\beta_{\text{time}}$ when using EBF for the spatial process with $L = 35$.}
  \label{fig:ebfpost}
\end{figure}

\begin{figure}  % markdown/fire-analysis/combine-tables.R
  \centering
  \includegraphics[width=\linewidth]{plots/gsk-post-betatime.pdf}
  \caption{Posterior summaries of $\beta_{\text{time}}$ when using GSK for the spatial process with $L = 35$.}
  \label{fig:gskpost}
\end{figure}

% \begin{figure}  % markdown/fire-analysis/combine-tables.R
%   \centering
%   \includegraphics[width=\linewidth]{plots/fire-bs-mean}
%   \caption{Average Brier score for exceeding q(0.95) (left). Average Brier score for exceeding q(0.99) (right).}
%   \label{fig:avgqscore}
% \end{figure}

% \begin{figure}  % markdown/fire-analysis/combine-tables.R
% 	\centering
% 	\includegraphics[width=\linewidth]{plots/fire-qs-mean}
% 	\caption{Average quantile score for q(0.95) (left). Average quantile score for q(0.99) (right).}
%   \label{fig:avgqscore}
% \end{figure}

% \begin{figure}  % markdown/fire-analysis/combine-tables.R
% 	\centering
% 	\includegraphics[width=0.47\linewidth]{plots/fire-timing}
% 	\caption{Timing comparison of basis functions to kernel functions for the spatial process (100 iterations)}
%   \label{fig:timingcompare}
% \end{figure}

Based upon the cross-validation results, we reran the full data analysis using $L = 15$ basis functions.
\hl{Figure here with panel of location \& scale: mean, sd, and P($\beta_t > 0$)}


\subsection{Model checking and sensitivity analysis}

\subsection{Analysis of annual precipitation}\label{s:precip}
We also conduct an analysis of the precipitation data presented in \citep{Reich2012}.
The data are climate model output from the North American Regional Climate Change Assessment Program (NARCCAP).
This data consists of $n_s = 697$ grid cells at a 50km resolution in the eastern US, and includes historical data (1969 -- 2000) as well as future conditions (2039 -- 2070).

\begin{figure}  % markdown/precipitation/cv-setup.R
  \centering
  \includegraphics[width=\linewidth]{plots/precip-ts}
  \caption{Time series of yearly max precipitation for current (1969 -- 2000) (left). Time series of yearly max precipitation for future (2039 -- 2070) (right).}
\end{figure}

\hl{Include figures of locations of grid cells}

\subsection{Results for precipitation analysis}\label{s:results-precip}

\begin{figure}  % markdown/precipitation/combine-tables.R
  \centering
  \includegraphics[width=\linewidth]{plots/precip-bs.pdf}
  \caption{Brier scores for current and future precipitation analysis.}
  \label{fig:precip-bs}
\end{figure}

\begin{figure}  % markdown/precipitation/combine-tables.R
  \centering
  \includegraphics[width=\linewidth]{plots/precip-qs.pdf}
  \caption{Quantile scores for current and future precipitation analysis.}
  \label{fig:precip-qs}
\end{figure}

\begin{figure}  % markdown/precipitation/combine-tables.R
  \centering
  \includegraphics[width=\linewidth]{plots/precip-post-time.pdf}
  \caption{Posterior distributions for $\beta_{\text{time}}$ for $\mu$ (left) and $\log(\sigma)$ (right).}
  \label{fig:precip-qs}
\end{figure}

\section{Conclusions}\label{s:con}
